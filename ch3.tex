% Copyright (c) 2023 Carl Martin Ludvig Sinander.

% This program is free software: you can redistribute it and/or modify
% it under the terms of the GNU General Public License as published by
% the Free Software Foundation, either version 3 of the License, or
% (at your option) any later version.

% This program is distributed in the hope that it will be useful,
% but WITHOUT ANY WARRANTY; without even the implied warranty of
% MERCHANTABILITY or FITNESS FOR A PARTICULAR PURPOSE. See the
% GNU General Public License for more details.

% You should have received a copy of the GNU General Public License
% along with this program. If not, see <https://www.gnu.org/licenses/>.

%%%%%%%%%%%%%%%%%%%%%%%%%%%%%%%%%%%%%%%%%%%%%%%%%%%%%%%%%%%%%%%%%%%%%%%



This chapter concerns repeated allocation problems.
In each period, the principal has one good, which she can either allocate to the agent or keep,
and the agent has a privately-known valuation $t$ for the good.
The principal wishes to allocate only to high types of the agent.
Every type of the agent wants the good, but high types want it more.

Here's an example: the principal is the `dean' of a university, and the agent is a department head. In each year, the department has one (best) job candidate, who may be either decent or excellent.
The dean only wants to hire excellent researchers.
The department (which does not internalise the cost to the university of hiring) always wants to hire, but is more keen when its candidate is excellent.

If the agent gets the good, then the agent and principal earn payoffs of $t$ and $t-c$, respectively; otherwise they both get zero.%
	\footnote{One interpretation is that the principal shares the agent's preferences, except that she also cares about the cost $c$ of producing/allocating the good.}
Both parties are expected-utility-maximisers who discount their future payoffs with discount factor $\delta \in (0,1)$.
Monetary transfers are unavailable.

Since there are a lot of periods, let's simplify by having only two types $t \in \{\ell,h\}$.
We assume that $0 < \ell < c < h$,
so that the principal wishes to allocate only to high types,
while all types of the agent want the good (though high types want it more).

Time will be indexed by $n \in \{0,1,\dots\}$.
The agent's random type process $( T_n )_{n=0}^\infty$ is a stationary Markov chain;
that just means that there are two fixed `transition probabilities' $u,d \in (0,1)$ (`up' and `down') such that
%
\begin{align*}
	\PP\left( T_{n+1} = h \middle|
	T_n = t_n, \dots,
	T_0 = t_0 \right)
	&= \PP\left( T_{n+1} = h \middle|
	T_n = t_n \right)
	\\
	&=
	\begin{cases}
		u		& \text{if $t_n = \ell$} \\
		1-d		& \text{if $t_n = h$.} 
	\end{cases}
\end{align*}
%
The initial type $T_0$ is drawn from the stationary distribution of the Markov chain: this is the distribution assigning probability $p \coloneqq u / (u+d)$ to $t=h$.
We write $\mu$ for its mean: $\mu = (1-p) \ell + p h$.

An important special case is when types are IID over time: $u=1-d=p$.
More generally, $u \leq 1-d$ means that the agent's type is persistent (non-negatively auto-correlated).
\emph{(In §\ref{sec:ch3:guohorner}, we shall assume persistence; in §\ref{sec:ch3:nocommitment} we shall assume IID.)}

In principle, $\mu$ can be on either side of $c$.
We shall focus on the more interesting case in which $\mu < c$, so that the principal wishes to allocate only if she manages to screen the agent.%
	\footnote{However, §\ref{sec:ch3:token} does not require this assumption.}

We shall also assume that patience $\delta$ is `high enough' to make screening possible: $\delta > \ell/\mu$.%
	\footnote{This assumption is also unnecessary in §\ref{sec:ch3:token}.}
This condition is easily interpreted in the IID case ($u=1-d=p$):
it ensures that a low type is willing to forego consumption today if promised that she'll be allowed to consume tomorrow no matter what ($\ell < \delta \mu$).

The analysis differs slightly depending on when the agent learns her type.
In §\ref{sec:ch3:token}, we shall assume that the agent learns the realisation of her entire type sequence $(T_n)_{n=0}^\infty$ at the outset (period $n=0$).
In §\ref{sec:ch3:guohorner}--§\ref{sec:ch3:nocommitment}, we shall instead assume that the agent learns the realisation of $T_n$ only at the dawn of period $n$, for each $n \in \{0,1,\dots,\}$.



%%%%%%%%%%%%%%%%%%%%%%%%%%%%%%%%%%%
%%%%%%%%%%%%%%%%%%%%%%%%%%%%%%%%%%%
\section{Quota mechanisms}
\label{sec:ch3:token}
%%%%%%%%%%%%%%%%%%%%%%%%%%%%%%%%%%%
%%%%%%%%%%%%%%%%%%%%%%%%%%%%%%%%%%%

Assume (mostly for simplicity) that the agent learns the realisation of her entire type sequence $(T_n)_{n=0}^\infty$ at the outset.

Here's a mechanism:
the agent freely chooses
the probability $q_n \in [0,1]$ with which she will consume the good in each period $n$,
subject to the lifetime budget constraint
%
\begin{equation*}
	(1-\delta) \sum_{n=0}^\infty \delta^n q_n 
	\leq K ,
\end{equation*}
%
where $K \in [0,1]$ is a constant.
We could call this a \emph{discounted quota:}
it caps the discounted total quantity that the agent may consume.

We could just as well replace this lifetime budget constraint with a dynamic budget constraint: the agent starts out with $M_0$ `tokens',
which may be spent to obtain the good and which accrue interest at rate $\delta^{-1}-1$ if saved:
%
\begin{equation*}
	q_n + M_{n+1}
	\leq \delta^{-1} M_n 
	\quad \text{in each period $n$.}
\end{equation*}
%
We impose the no-Ponzi-scheme condition
%
\begin{equation*}
	\lim_{n \to \infty} \delta^n M_n = 0 .
\end{equation*}
%
Then we have
%
\begin{align*}
	M_0
	&\geq \delta M_1 + \delta q_0
	\\
	&\geq \delta ( \delta M_2 + \delta q_1 )
	+ \delta q_0
	= \delta^2 M_2
	+ \delta ( q_0 + \delta q_1 )
	\\
	&\geq \delta^2 ( \delta M_3 + \delta q_2 )
	+ \delta ( q_0 + \delta q_1 )
	= \delta^3 M_3
	+ \delta \left( q_0 + \delta q_1 + \delta^2 q_2 \right)
	\\
	&\;\; \vdots
	\\
	&\geq \lim_{n \to \infty} \delta^n M_n
	+ \delta \sum_{n=0}^\infty \delta^n q_n 
	= \delta \sum_{n=0}^\infty \delta^n q_n ,
\end{align*}
%
or
%
\begin{equation*}
	(1-\delta) \sum_{n=0}^\infty \delta^n q_n
	\leq (1-\delta) \delta^{-1} M_0
	\eqqcolon K.
\end{equation*}

The principal's first-best allocation
is the one in which the agent consumes when and only when her type is high.
The first-best discounted total quantity is therefore
%
\begin{equation*}
	Q_\delta \coloneqq (1-\delta)
	\sum_{n=0}^\infty \delta^n \1_{\{h\}}(T_n) .
\end{equation*}
%
Its expected value is
%
\begin{equation*}
	\E( Q_\delta )
	= (1-\delta) \sum_{n=0}^\infty \delta^n \E\left( \1_{\{h\}}(T_n) \right)
	= (1-\delta) \sum_{n=0}^\infty \delta^n p
	= p .
\end{equation*}
%
Given this, a natural choice of cap $K$ is $K = p$.%
	\footnote{I am \emph{not} claiming that among discounted-quota mechanisms, the one with $K=p$ gives the principal the highest payoff. That's not true in general.}

The incentive properties of discounted quotas are simple:
the mechanism leaves the agent no scope to obtain a larger discounted total quantity.
The only question is thus in \emph{which} periods the agent consumes,
and this is a dimension along which incentives are aligned:
both parties prefer for the agent to consume when her type is high.

The mechanism isn't perfect, however, since the principal's first-best discounted total quantity
%
\begin{equation*}
	Q_\delta = (1-\delta) \sum_{n=0}^\infty \delta^n \1_{\{h\}}(T_n)
\end{equation*}
%
is random, not always equal to its expectation $p$.
If the agent's type is high for several period in a row early on, then the first-best discounted total quantity is greater than $p$,
but the agent lacks the tokens needed to consume more than $p$.
Conversely, if her type is initially low for several periods, then she accrues enough saved tokens to be able to purchase the good in some periods in which her type is low.

This problematic variability of $Q_\delta$ is less severe the greater is the discount factor $\delta$.
To see why,
assume for simplicity that the agent's type is IID over time ($u=1-d=p$),
and calculate
%
\begin{align*}
	\Var( Q_\delta )
	&= (1-\delta)^2 \sum_{n=0}^\infty
	\delta^{2n} \Var\left( \1_{\{h\}}(T_n) \right)
	\\
	&= (1-\delta)^2 \sum_{n=0}^\infty
	\delta^{2n} p (1-p)
	\\
	&= \frac{ (1-\delta)^2 }{ 1 - \delta^2 } p (1-p)
	= \frac{ (1-\delta)^2 }{ (1-\delta) (1+\delta) } p (1-p)
	= \frac{ 1-\delta }{ 1+\delta } p (1-p) .
\end{align*}
%
By inspection, $\Var( Q_\delta )$ vanishes as $\delta$ approaches $1$.
This suggests that we may expect $Q_\delta$ to be close to $p$
with high probability if $\delta$ is large.
That is indeed the case: by invoking Chebyshev's inequality, we obtain
%
\begin{equation*}
	\PP\left( \abs*{ Q_\delta - p } \geq \eps \right)
	\leq \frac{ \Var( Q_\delta ) }{ \eps^2 } 
	\quad \text{for any $\eps>0$,}
\end{equation*}
%
so that
%
\begin{equation*}
	\lim_{\delta \uparrow 1 } \PP( \abs*{ Q_\delta - p } \geq \eps )
	= 0
	\quad \text{for any $\eps>0$.}
\end{equation*}
%
(In other words, $Q_\delta$ \emph{converges in probability} to $p$ as $\delta \uparrow 1$.
Such a result is called a \emph{weak law of large numbers.})

The discounted-quota mechanism with cap $K=p$
therefore provides close to the first-best discounted total quantity
with high probability when $\delta$ is high.
This suggests (but does not prove!) that if $\delta$ is large, then this discounted-quota mechanism gives the principal a \emph{payoff} close to her first-best payoff (which is $(h-c) Q_\delta$, by the way---why?).
In other words, this mechanism is \emph{approximately optimal} (for large $\delta$). Formally:

\begin{proposition}
	%
	\label{proposition:JacksonSonnenschein}
	%
	Assume that the agent's type is IID over time ($u=1-d=p$)
	and that the agent learns the realisation of her entire type sequence $(T_n)_{n=0}^\infty$ at the outset.
	Let $U_\delta^K$ denote the principal's (ex-post, random) payoff under the discounted-quota mechanism with cap $K \in [0,1]$
	when the discount factor is $\delta \in (0,1)$.
	Then $(h-c) Q_\delta - U_\delta^p \to 0$ a.s. as $\delta \uparrow 1$.%
		\footnote{Hence by the bounded convergence theorem,
		the \emph{expected} payoff $\E( U_\delta^p )$ converges as $\delta \uparrow 1$
		to the first-best expected payoff $(h-c)p$.}
	%
\end{proposition}

The proof relies on a simple \emph{strong law of large numbers.}
The IID assumption can be relaxed by using a fancier strong law of large numbers.

\begin{proof}
	%
	Fix a cap $K \in [0,1]$ and a discount factor $\delta \in (0,1)$.
	Clearly the agent always exhausts the budget.
	If $Q_\delta \geq K$, then she consumes only when $T_n=h$,
	so her payoff is $h K$.
	If $Q_\delta < K$, then the agent consumes whenever $T_n=h$,
	and spends the rest of the budget when $T_n=\ell$;
	thus her payoff is $h Q_\delta + \ell (K-Q_\delta)$.
	The principal's payoff is the agent's payoff minus $cK$, so
	%
	\begin{multline*}
		(h-c) Q_\delta
		- U_\delta^K
		\\
		\begin{aligned}
			&= (h-c) Q_\delta
			- \Bigl[
			h \min\{K,Q_\delta\}
			+ \ell \max\{K-Q_\delta,0\}
			- c K 
			\Bigr]
			\\
			&= (h-c) Q_\delta
			- \Bigl[
			- h \max\{-K,-Q_\delta\}
			+ \ell \max\{K-Q_\delta,0\}
			- c K 
			\Bigr]
			\\
			&= (h-c) Q_\delta
			- \Bigl[
			- (h-\ell) \max\{K-Q_\delta,0\}
			+ (h-c)K
			\Bigr]
			\\
			&= - (h-c) ( K - Q_\delta )
			+ (h-\ell) \max\{K-Q_\delta,0\} .			
		\end{aligned}
	\end{multline*}
	%
	For $K=p$,
	as $\delta \uparrow 1$,
	the right-hand side vanishes a.s.
	since $Q_\delta \to p$ a.s.
	by a strong law of large numbers for discounted sums \parencite[see][]{Lai1974}.
	%
\end{proof}


\begin{remark}
	%
	\label{remark:het}
	%
	Zooming out, the model is one in which several decisions must be taken;
	we may re-interpret these as distinct, simultaneously-chosen actions.
	The principal and agent are still assumed to have payoffs additive across decisions, but the weights assigned to different decisions need not have the geometric form $1, \delta, \delta^2, \dots$.
	As above, the result is that `linking' many different decisions via a weighted quota can yield good outcome.
	%
\end{remark}


\paragraph{The literature.}
\textcite{JacksonSonnenschein2007} showed that the principal's first-best can be approximated by `linking' the allocations across different periods via something like a discounted quota.
(The theorem in this paper is true as stated, but the proof is wrong: see \textcite{BallJacksonKattwinkel2022} for the correction.)
\textcite{MatsushimaMiyazakiYagi2010,Ishii2016} go further, characterising the full set of allocations that are (virtually) implementable via quota mechanisms.
(The characterisation: virtual implementability via quotas is equivalent to implementability using monetary transfers.)
\textcite{Frankel2016jet} is a clear-eyed and general formulation of the quotas idea, with good references references to related work.
\textcite{Frankel2016aej} provides a similarly insightful and general treatment of the idea in \Cref{remark:het}.
(None of these papers considers precisely the model above,
and the mechanism in Jackson--Sonnenschein is not quite a discounted-quota mechanisms.)



%%%%%%%%%%%%%%%%%%%%%%%%%%%%%%%%%%%
%%%%%%%%%%%%%%%%%%%%%%%%%%%%%%%%%%%
\section{Optimal mechanisms with commitment}
\label{sec:ch3:guohorner}
%%%%%%%%%%%%%%%%%%%%%%%%%%%%%%%%%%%
%%%%%%%%%%%%%%%%%%%%%%%%%%%%%%%%%%%

In this section, we'll assume that the agent's valuation is persistent ($u \leq 1-d$),
and that the agent learns her type in real time (she observes the realisation of $T_n$ only at the dawn of period $n$).

Discounted quota mechanisms ensure IC
precisely by capping the discounted total quantity.
This inflexibility causes inefficiency, since the first-best discounted total quantity $Q_\delta$ is random.
Is it possible to improve by supplying a greater discounted total quantity when the agent's type turns out to be high in several early periods, while preserving IC?

Consider the case of two periods, $n \in \{0,1\}$, 
with IID types ($u = 1-d = p$),
and let's neglect discounting ($\delta=1$).
In the mechanism with cap $K=1$,
the agent consumes in period $n=0$ if her type is high ($T_0=h$),
and otherwise waits and consumes in period $n=1$. (Why?)
The principal's payoff is therefore
%
\begin{equation*}
	p h + (1-p) \mu - c.
\end{equation*}
%
(We're interested in parameters for which this is the best quota mechanism.%
	\footnote{By $\mu<c$, the principal strictly prefers $K=0$ to $K=2$.
	She strictly prefers $K=1$ to $K=0$ iff $p h + (1-p) \mu > c$.}%
)
Phrased differently, the $n=0$ allocation is efficient,
while at $n=1$ an agent who was high at $n=0$ gets nothing
and an agent who was low at $n=0$ gets the good for sure.
The principal doesn't like handing out the good in period $1$ without screening (since $\mu < c$).
So let's lower the probability $e$ with which this occurs
as much as possible, subject to preserving the period-$0$ low type's incentive to forego consumption; this yields $e = \ell / \mu$.
This mechanism does not cap the total quantity supplied; rather, it provides a smaller total quantity when the agent's period-$0$ type is low.


To extend this idea to the full infinite-horizon model with discounting,
we introduce \emph{entitlement mechanisms:}
the agent has an \emph{entitlement} to claim the good for a certain number of consecutive periods `no questions asked'.
Whenever she foregoes the opportunity to obtain the good, her entitlement is augmented.

A little more fully, the agent begins each period with some entitlement $E \in [0,\infty]$.
If $E>0$, then she may claim the good, in which case her entitlement drops to $E-1$.
(If $E \in (0,1)$, then she gets the good with probability $E$, and her entitlement drops to zero.)
If she does not claim the good, then her entitlement rises by some history-dependent amount, possibly to $\infty$.%
	\footnote{In a period in which the entitlement rises to $\infty$, the agent may in fact also receive the good with positive probability.}

By appropriately choosing the amount by which the agent's entitlement is augmented when she does not claim the good, it is possible to ensure that as long as $E \in (0,\infty)$, the agent is (only just) willing to forego consumption when her type is low. (This clearly implies that she will claim the good when her type is high.)
In other words, allocation is exactly efficient until $E$ hits $0$ or $\infty$.
(So inefficiencies are back-loaded.)

The agent's entitlement will eventually hit either $0$ or $\infty$.%
	\footnote{In finite time, almost surely; this follows from the Borel--Cantelli lemma.}
We might call the former `termination'
and the latter `tenure'.
The discounted total quantity supplied is evidently not constant (by contrast with discounted-quota mechanisms).


\paragraph{The literature.}
\textcite{GuoHorner2020} study this problem, and prove that entitlement mechanisms are optimal.%
	\footnote{They prove this under an additional `enough-patience' assumption: $\delta > 1/2$.}
You'll learn a lot by reading it.
The technique they use to solve for an optimal mechanism is that of \textcite{SpearSrivastava1987}, which is well-worth learning if you're interested in dynamic mechanism design with full commitment and no transfers.
It offers a general recursive way of formulating dynamic contracting problems (with the agent's promised utility as the state variable).
(It's a standard enough topic that you can find lecture notes on it; I don't know of a textbook treatment, however. It is most commonly applied to dynamic models of \emph{moral hazard,} rather than adverse selection.)



%%%%%%%%%%%%%%%%%%%%%%%%%%%%%%%%%%%
%%%%%%%%%%%%%%%%%%%%%%%%%%%%%%%%%%%
\section{Optimal mechanisms without commitment}
\label{sec:ch3:nocommitment}
%%%%%%%%%%%%%%%%%%%%%%%%%%%%%%%%%%%
%%%%%%%%%%%%%%%%%%%%%%%%%%%%%%%%%%%

In this section, we'll assume that the agent's valuation is IID over time ($u = 1-d = p$),
and that the agent learns her type in real time (she observes the realisation of $T_n$ only at the dawn of period $n$).
We'll also add an extra `enough patience' assumption:
%
\begin{equation*}
	(1-\delta) (c-\mu)
	\leq \delta \left( 1 - \frac{c}{\ell} \right) \left( \mu - \ell \right) .
\end{equation*}

Suppose that the principal lacks the ability to commit over time.
She can take certain actions (or `commitments', if you prefer) within a given period, but cannot bind her future self.
What can she achieve?

This is analytically a very different sort of question. The reason is that the principal now also has IC constraints: in each period, the principal must (given her belief) find it optimal to take the action specified by the mechanism.

A different way of thinking about limited commitment is that absent commitment, the agent and principal simply play a game: in particular, a repeated game of incomplete information, with the private information on one side only.
Each equilibrium of this game induces some allocation,%
	\footnote{Note that equilibria can differ off the path of play, but still induce the same allocation.}
and optimal incentive-provision from the principal's perspective
amounts to playing a principal-preferred equilibrium.

It remains to specify what the principal can commit to (within a period).
We'll assume that she may commit to a within-period delegation mechanism:
she either offers the good to the agent or not,
and if she offers it then the agent may accept or refuse.
This gives the principal the ability to commit not to allocate (within a given period), which is potentially valuable.


The first thing to check is whether relaxing commitment matters:
if the entitlement mechanism that was optimal under full commitment is an equilibrium outcome, then we'd be done!
But it's not an equilibrium outcome.
To see why, consider a history at which the agent's entitlement $E$ has hit $\infty$. (This occurs with positive probability.)
The agent is now permitted to consume the good in every subsequent period, despite the fact that the principal prefers \emph{not} to allocate absent screening ($\mu < c$).
Clearly the principal strictly prefers to deviate to never allocate again.

So what \emph{are} equilibrium outcomes?
And which is the principal's favourite?


\paragraph{The literature.} \textcite{LipnowskiRamos2020} characterise principal-optimal equilibria (and more generally, Pareto-optimal equilibria).%
	\footnote{In their model, the agent learns $T_n$ only if and when she is offered the good in period $n$. This does not make a difference, I believe.
	For suppose the principal-preferred equilibrium were to feature the agent conditioning her behaviour in period $n+1$
	on what $T_n$ if she was not offered the good in period $n$.
	Since $T_n$ has no statistical effect on $T_{n+1}$ (that's our IID assumption),
	this just amounts to the agent randomising in period $n+1$ (when the principal did not offer the good in period $n$);
	so we can replicate this behaviour by having the agent mix (i.e. she can use her own set of dice to randomise, instead of conditioning her behaviour on the `natural dice' $T_n$).}
Their analysis relies on the `APS' technique \parencite{AbreuPearceStacchetti1990}, which provides a recursive formulation of repeated games with imperfect monitoring. This is a standard (indeed, essential!) tool for analysing repeated interactions without commitment.
The original APS paper is rather beautiful,
but you may prefer to learn it from the textbook treatment of \textcite{MailathSamuelson2006}, or from whatever lecture notes you can find around the internet.
Another nice paper on dynamic allocation without commitment is \textcite{DeclippelEliazFershtmanRozen2021}, who study a model with \emph{two} agents.
